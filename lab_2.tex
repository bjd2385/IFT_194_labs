\documentclass[leqno, 11pt]{article}

\usepackage{lmodern}
\usepackage[scaled]{beramono}
\usepackage[T1]{fontenc}
\usepackage{amssymb}
\usepackage{amsmath}
\usepackage{hyperref}
\hypersetup{%
  colorlinks=true,
  linkcolor=magenta,
  filecolor=magenta,
  urlcolor=magenta
}

\usepackage[margin=1in]{geometry}
\usepackage{listings}
\usepackage{graphicx}
\usepackage{xcolor}
\usepackage{caption}

\captionsetup{%
  width=1.0\linewidth,
  justification=centering
}

\usepackage{verbatimbox}
\usepackage{siunitx}
\usepackage{colortbl}
\usepackage[most]{tcolorbox}

% TODO: Set up captions with tcolorbox so the user doesn't have to explicitly
%       pass a caption={} to each \lstlisting command.
\newtcolorbox[blend into=figures]{codefigure}[1][]{%
  enhanced jigsaw, 
  float=!ht, 
  breakable, 
  frame hidden, 
  colback=white,
  #1
}

% Better `@' symbol
\newcommand{\at}{\mbox{}{\fontfamily{ptm}\selectfont @}\mbox{}}

\newcommand\blfootnote[1]{%
  \begingroup
    \renewcommand\thefootnote{}\footnote{#1}
    \addtocounter{footnote}{-1}
  \endgroup
}

%\graphicspath{"/home/brandon/Desktop/IFT_194/labs/photos"}

\definecolor{tableheaderrow}{HTML}{CED3DB}

\definecolor{javacommentscolor}{HTML}{646464}
\definecolor{javakeywordscolor}{HTML}{7F0055}
\definecolor{javastringscolor}{HTML}{2A00FF}

\lstset{%
  basicstyle=\scriptsize\ttfamily, % code to be displayed as monospace
  breaklines=true,
  %frame=b
  commentstyle=\color{javacommentscolor},
  keywordstyle=\color{javakeywordscolor},
  stringstyle=\color{javastringscolor},
  showstringspaces=false,  % do not show string spaces character
  tabsize=4,  % change tabs to spaces
  keywordsprefix={@},  % capture method annotations and doctools
  %showtabs=true,
  %tab=|
}

% TODO: Modify listings in the next lab so that I don't have to wrap in figure
%       env anymore -- makes the following lines irrelevant
\renewcommand{\lstlistingname}{Figure}

\newcommand{\centeredimage}[2]{%
  \begin{center}
    \includegraphics[scale=#1]{#2}
  \end{center}
}

\title{\vspace{6ex}Fundamental Programming Structures in Java\\
  \Large IFT 194: Lab 2}
\author{Brandon Doyle\\
\href{mailto:bdoyle@asu.edu}{bdoyle5\at{}asu.edu}\\
1215232174\\[1em]
Dr. Usha Jagannathan\\
\href{mailto:Usha.Jagannathan@asu.edu}{Usha.Jagannathan\at{}asu.edu}}

\setlength{\parindent}{0em}
\setlength{\parskip}{0.5em}

\makeatletter
% Top-align float-only pages
% https://tex.stackexchange.com/a/28565/96382
\setlength{\@fptop}{0pt}
\setlength{\@fpbot}{0pt plus 1fil}

% Share Figure and listings counter
\AtBeginDocument{%
  \let\c@figure\c@lstlisting
  \let\thefigure\thelstlisting
  \let\ftype@lstlisting\ftype@figure
}
\makeatother

\begin{document}
\begin{titlepage}
\clearpage\maketitle
\thispagestyle{empty}
\end{titlepage}
\blfootnote{View the source of this document on \href{https://github.com/bjd2385/IFT_194_labs/blob/master/\jobname.tex}{GitHub}.}
\section*{Pre-Lab Exercises}
\subsection*{A. Textbook Sections 5.1--5.3}
\begin{enumerate}
  \item We are tasked with rewriting various conditions in valid Java syntax. 
        \begin{enumerate}
          \item The condition \texttt{x > y > z} may be written in Java as \texttt{x > y \&\& y > z}, i.e. we need to join the two comparisons by the $\wedge$--logical operator. This is a result of the type of objects the relational operators act upon; because \texttt{x > y} returns a \texttt{boolean} type, we receive a compile-time error (invalid types).
    
                Interestingly enough, this \textit{is} valid Python syntax due its recursive \href{https://github.com/python/cpython/blob/master/Grammar/Grammar#L93}{comp\_op} Grammar definition, so we may (hypothetically) write an inifinite sequence \texttt{expr comp\_op ... expr comp\_op expr}. $\wedge$--logical operators are automatically inserted.
              \item The statement ``x and y are both less than 0'' may quite simply be expressed as \texttt{x < 0 \&\& y < 0}.
              \item The statement ``neither x nor y are less than 0'' may be expressed as \texttt{x >= 0 \&\& y >= 0}, or the negation of the previous predicate, i.e. \texttt{!(x < 0 \&\& y < 0)}. I think the former is more readable, however.
              \item The statement ``x equals y but not z'' may be written as \texttt{x == y \&\& x != z}.
        \end{enumerate}
  \item We are tasked with writing an \texttt{if--then} statement to state whether a student has made the Dean's list. Please see \autoref{fig:one} for my solution.
  \item We are tasked with completing/fixing an example program that computes the raise an employee will receive based on their performance value. Please see \autoref{fig:two} for my solution.
\end{enumerate}
\section*{Textbook Section 5.4}
\begin{enumerate}
  \item Suppose we have a loop as follows.
        \begin{figure}[h!]
          \centering
          \lstinputlisting[language=java,xleftmargin=0.25\textwidth]{%
            /home/brandon/eclipse-workspace/ift_194_labs/src/lab_2/SimpleLoop.java}
          \caption{SimpleLoop.java.}
          \label{fig:three}
        \end{figure}\\
        This program outputs the sequence \texttt{1..10} when it's executed. Reversing the order of the statements \texttt{count++} and \texttt{System.out.println(count + `` '')} will thus print the sequence \texttt{2..11}. This is the case because we are then incrementing each value in \texttt{1..10} \textit{prior} to printing. 
        
        As a quick comparison, because I think it's awesome how some languages (or paradigms) are better at expressing certain concepts than others, I've written the same program in Haskell with monads in \autoref{fig:four}. (I think most of the complexity is introduced by immutability.)
  \item Next we will consider the following program/loop.
        \begin{figure}[h!]
          \centering
          \lstinputlisting[language=java,xleftmargin=0.25\textwidth]{/home/brandon/eclipse-workspace/ift_194_labs/src/lab_2/TraceLoop.java}
          \caption{TraceLoop.java.}
          \label{fig:five}
        \end{figure}\\
        When run, the output of this program is the statement ``The sum of your integers is 20.'' Below is a table containing the values of \texttt{count}, \texttt{sum}, and \texttt{nextVal} as this program is executing.
        \begin{center}
          \begin{tabular}{|c|c|c|}
            \hline
            \rowcolor{tableheaderrow}
            \texttt{count} & \texttt{sum} & \texttt{nextVal}\\
            \hline
            1 & 0  & 2  \\
            \hline
            2 & 2  & 4  \\
            \hline
            3 & 6  & 6  \\
            \hline
            4 & 12 & 8  \\
            \hline
            5 & 20 & 10 \\
            \hline
          \end{tabular}
        \end{center}
        If we wished instead to add the first \textit{count} nonzero integers, we could modify the \texttt{count} variable to start at some positive integer, such as \texttt{5}, and the \texttt{while}-loop's predicate to instead state \verb|count-- > 0| (while removing the following  \texttt{count++}), which outputs the following.
        \begin{center}
          \begin{tabular}{|c|c|c|}
            \hline
            \rowcolor{tableheaderrow}
            \texttt{count} & \texttt{sum} & \texttt{nextVal}\\
            \hline
             4 & 0  & 2  \\
            \hline
             3 & 2  & 4  \\
            \hline
             2 & 6  & 6  \\
            \hline
             1 & 12 & 8  \\
            \hline
             0 & 20 & 10 \\
            \hline
            -1 & 30 & 12 \\
            \hline
          \end{tabular}
        \end{center}
  \item We're tasked next with writing a loop that will print the statement ``I love computer science!!'' 100 times. See \autoref{fig:six} for my solution. Also, this type of loop is count-controlled, and I would much rather have used a \texttt{for}-loop.
  \item In this question we're asked to write a program that accepts integers from the user and adds them all up. See \autoref{fig:seven} for my solution.

        Also note that I've used a regular expression to match user input -- this loop is \textit{not} ``count-controlled,'' it will exit only if the user decides to terminate the loop. The following is an example input.
        \begin{verbbox}[\mbox{}\scriptsize]
Enter the next integer: 1
Intermediate total: 1
Keep going? [y|n]: y
Enter the next integer: 13
Intermediate total: 14
Keep going? [y|n]: 7
Total: 14
Input integers: 2
*** Exiting
        \end{verbbox}
        \begin{figure}[h!]
          \centering
          \theverbbox
        \end{figure}
      \item We are asked to write a loop that counts backward from 10 to 1. The code provided has two issues, namely that the \texttt{while}-loop's predicate should be \texttt{count > 0} (we'd like to stop at 1, not 0), and the variable \texttt{count} should be decremented, not incremented, as \verb|count--|. See \autoref{fig:eight} for my solution.
\end{enumerate}
\section*{Two Meanings of Plus}
In Java the (\texttt{+}) operator has the ability to work on numbers and strings. This is a form of polymorphism, because the way the code works depends upon the the type of data it's acting upon.

\begin{enumerate}
  \item As an example, we're given the program in \autoref{fig:nine} to run. The output is as follows.
        \begin{verbbox}[\mbox{}\scriptsize]
This is a long string that is the concatenation of two shorter strings.
The first computer was invented about55years ago.
8 plus 5 is 85
8 plus 5 is 13
13 equals 8 plus 5.
        \end{verbbox}
        \begin{figure}[h!]
        \centering
        \theverbbox
        \end{figure}\\
        In the first statement, the two strings are concatenated. In the second, the strings and number are concatenated, but space wasn't provided at the end of the former string, and at the beginning of the second, so it's a bit ``scrunched.'' In the third, both integers were cast to \texttt{String}s and concatenated to the end; however, in the forth parentheses were added, so the expression was evaluated prior to being cast and concatenated. Lastly, no parentheses were necessary to evaluate the leading expression because the argument is evaluated left-to-right.
  \item Please see \autoref{fig:ten} for my solution to this problem (it's quite simple).
\end{enumerate}
\section*{Painting a Room}
In this section we're presented with the problem of prompting a user for the dimensions of a room, the walls of which we would like to paint. However, each gallon of paint will only cover 350 \si{ft^{2}}, and we would like to know the amount of paint we need to buy. See \autoref{fig:eleven} for my initial solution. An example session is as follows.
\begin{verbbox}[\mbox{}\scriptsize]
Please enter an integer for the length: 5
Please enter an integer for the width: 5
Please enter an integer for the height: 10
L, W, H: 5, 5, 10
Gallans of paint required: 0.5714285714285714
\end{verbbox}
\begin{figure}[h!]
  \centering
  \theverbbox
\end{figure}\\
Recall that the area of the walls is given by
\begin{align}
  \underbrace{2\times H\times L}_{\text{adding 2 walls length-wise}} + \underbrace{2\times H\times W}_{\text{two walls width-wise}}&=2\times H\times\left(L + W\right).\notag
\end{align}
Hence, working out the example above, we have
\begin{align}
  \dfrac{2\times 10\times \left(5 + 5\right)}{350}&=\dfrac{20\times\left(10\right)}{350}\notag\\&=\dfrac{200}{350}\notag\\&=0.57...\notag
\end{align}
which is exactly my program's output.

Now, suppose we wanted to subtract windows and doors? Assuming they have areas of 15 \si{ft^{2}} and 20 \si{ft^{2}}, respectively, we can add this functionality by adding two more field variables, local variables, and calls to \texttt{getAttr}, in addition to modifying the equation to
\begin{align}
  2\times H\times L + 2\times H\times W \color{red}{- \Omega\times 15 - D\times 20}&=2\times H\times\left(L + W\right)\color{red}{- \Omega\times 15 - D\times 20}.\notag
\end{align}
To save space in this document I've submitted a gunzipped tarball with all of my code. Another command that may be of use if you're using Linux is to run \texttt{diff -y -W 150 code/Paint* | less}, which  will display relevant changes side-by-side (or simply cat the included differences.txt file).
\section*{Counting Characters}
In this section we're tasked with writing a program that will accept a string via a \texttt{Scanner} object from the user and count certain characters in the string. See \autoref{fig:twelve} for my solution and \autoref{fig:thirteen} for an example session on the following page.
\begin{verbbox}[\mbox{}\scriptsize]
Character Counter
=================

Please enter a string of text: hello, world! I enjoy writing JAva

Number of blank spaces: 5
Number of [aA]'s: 2
          [eE]'s: 2
          [sS]'s: 0
          [tT]'s: 1

Add another string? [y|n]: y

Please enter a string of text: of course yo

Number of blank spaces: 2
Number of [aA]'s: 0
          [eE]'s: 1
          [sS]'s: 1
          [tT]'s: 0

Add another string? [y|n]: quit

*** Exiting
\end{verbbox}
\begin{figure}[t!]
  \centering
  \theverbbox
  \caption{Example session with my program in \autoref{fig:twelve}.}
  \label{fig:thirteen}
\end{figure}
\section*{Tracking Sales}
In this section we're concerned with tracking the number of sales of salespersons and printing various statistics about the data as it's entered by a user. See figure \autoref{fig:fourteen} for my solution, and \autoref{fig:fifteen} for another example session on the following page.
\begin{verbbox}[\mbox{}\scriptsize]
Enter the number of salespeople: 3

Enter sale for a salesperson 1: 5
Enter sale for a salesperson 2: 3
Enter sale for a salesperson 3: 8

Salesperson sales
-----------------
1    5
2    3
3    8

Total sales: 16
Average sale: 5.333333333333333
The max sale was by salesperson 3, which was 8
The minimum sale was by salesperson 2, which was 3

Enter a cutoff value (int): 4

Salesperson 1 exceeded the cutoff with 5 sales
Salesperson 3 exceeded the cutoff with 8 sales

Total salespeople exceeding the cutoff amount: 2
\end{verbbox}
\begin{figure}[h!]
  \centering
  \vspace{0.5cm}
  \theverbbox
  %% Manually specified figure number because I haven't worked out the details of
  %% the counters yet and how tcolorbox env interacts with them.
  \caption{Example session with my program in \hyperref[fig:fourteen]{Figure 15}.}
  \label{fig:fifteen}
\end{figure}
\section*{Conclusion}
In this lab I learned a lot more about Java. For instance, firstly, I learned that there are many of the same concepts in Java as there are in other OO programming languages, like Python. One such feature is context managers; as demonstrated in \autoref{fig:two}, Java too has the ability to implicitly call a \texttt{.close()} method on objects that implement the \texttt{AutoCloseable} interface. I also had a lot of fun working with more complex control structures, like loops and switch statements. I look forward to learning more about procedural and OO programming concepts.
\newpage
\begin{figure}
  \centering
  \lstinputlisting[language=java]{%
    /home/brandon/eclipse-workspace/ift_194_labs/src/lab_2/DeansList.java}
  \caption{DeansList.java. I decided to turn this program into a super simple command line utility to test the usage of \texttt{args} in the \texttt{main} function.}
  \label{fig:one}
\end{figure}
\begin{figure}
  \centering
  \lstinputlisting[language=java]{%
    /home/brandon/eclipse-workspace/ift_194_labs/src/lab_2/Salary.java}
  \caption{Salary.java. See also the documentation on \href{https://docs.oracle.com/javase/10/docs/api/java/lang/AutoCloseable.html}{AutoCloseable}, which provides a nice interface for closing files like Python's \href{https://docs.python.org/3/reference/compound_stmts.html\#with}{context managers}.}
  \label{fig:two}
\end{figure}
\begin{figure}
  \centering
  \lstinputlisting[language=haskell]{/home/brandon/Desktop/IFT_194/labs/code/increment.hs}
  \begin{verbatim}
$ ghc --make increment.hs
$ ./increment
1 2 3 4 5 6 7 8 9 10
2 3 4 5 6 7 8 9 10 11
  \end{verbatim}
  \caption{increment.hs. See question 1 in Textbook Section 5.4's questions above.}
  \label{fig:four}
\end{figure}
\begin{figure}
  \centering
  \lstinputlisting[language=java]{%
    /home/brandon/eclipse-workspace/ift_194_labs/src/lab_2/CSLoop.java}
  \caption{CSLoop.java.}
  \label{fig:six}
\end{figure}
\begin{figure}
  \centering
  \lstinputlisting[language=java]{%
    /home/brandon/eclipse-workspace/ift_194_labs/src/lab_2/Loop.java}
  \caption{Loop.java.}
  \label{fig:seven}
\end{figure}
\begin{figure}
  \centering
  \lstinputlisting[language=java]{%
    /home/brandon/eclipse-workspace/ift_194_labs/src/lab_2/Decrement.java}
  \caption{Decrement.java.}
  \label{fig:eight}
\end{figure}
\begin{figure}
  \centering
  \lstinputlisting[language=java]{%
    /home/brandon/eclipse-workspace/ift_194_labs/src/lab_2/Add.java}
  \caption{Add.java.}
  \label{fig:nine}
\end{figure}
\begin{figure}
  \centering
  \lstinputlisting[language=java]{%
    /home/brandon/eclipse-workspace/ift_194_labs/src/lab_2/MyOwnAdd.java}
  \caption{MyOwnAdd.java.}
  \label{fig:ten}
\end{figure}
\begin{figure}
  \centering
  \lstinputlisting[language=java]{%
    /home/brandon/eclipse-workspace/ift_194_labs/src/lab_2/Paint.java}
  \caption{Paint.java.}
  \label{fig:eleven}
\end{figure}
\begin{figure}
  \centering
  \lstinputlisting[language=java]{%
    /home/brandon/eclipse-workspace/ift_194_labs/src/lab_2/Count.java}
  \caption{Count.java.}
  \label{fig:twelve}
\end{figure}
% Break this code across multiple pages -- see TODOs above
\begin{codefigure}
  \label{fig:fourteen}
  \addtocounter{figure}{-1}
  \lstinputlisting[language=java, caption={Sales.java}, captionpos=b]{%
    /home/brandon/eclipse-workspace/ift_194_labs/src/lab_2/Sales.java}
\end{codefigure}
\end{document}
