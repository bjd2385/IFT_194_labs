\documentclass[leqno, 11pt]{article}

\usepackage{lmodern}
\usepackage[scaled]{beramono}
\usepackage[T1]{fontenc}
\usepackage{amssymb}
\usepackage{amsmath}
\usepackage{hyperref}
\hypersetup{%
  colorlinks=true,
  linkcolor=magenta,
  filecolor=magenta,
  urlcolor=magenta
}

\usepackage[margin=1in]{geometry}
\usepackage{listings}
\usepackage{graphicx}
\usepackage{caption}

\captionsetup{%
  width=1.0\linewidth,
  justification=centering
}

\usepackage{enumitem} % allow characters instead of numbers
\usepackage{verbatimbox}  % center \verbatim env in figure env
\usepackage{siunitx}  % SI unit formatting
\usepackage{colortbl}  % color tabular rows/columns
\usepackage[most]{tcolorbox}

% TODO: Set up captions with tcolorbox so the user doesn't have to explicitly
%       pass a caption={} to each \lstlisting command.
\newtcolorbox[blend into=figures]{codefigure}[1][]{%
  enhanced jigsaw, 
  float=!ht, 
  breakable, 
  frame hidden, 
  colback=white,
  #1
}

% Better `@' symbol
\newcommand{\at}{\mbox{}{\fontfamily{ptm}\selectfont @}\mbox{}}

\newcommand\blfootnote[1]{%
  \begingroup
    \renewcommand\thefootnote{}\footnote{#1}
    \addtocounter{footnote}{-1}
  \endgroup
}

%\graphicspath{"/home/brandon/Desktop/IFT_194/labs/photos"}

\usepackage{xcolor}

\definecolor{tableheaderrow}{HTML}{CED3DB}

\definecolor{javacommentscolor}{HTML}{646464}
\definecolor{javakeywordscolor}{HTML}{7F0055}
\definecolor{javastringscolor}{HTML}{2A00FF}

\definecolor{background}{HTML}{F2F7FF}

\lstset{%
  basicstyle=\linespread{0.8}\scriptsize\ttfamily, % code to be displayed as monospace
  breaklines=true,
  commentstyle=\color{javacommentscolor},
  keywordstyle=\color{javakeywordscolor},
  stringstyle=\color{javastringscolor},
  showstringspaces=false,  % do not show string spaces character
  tabsize=4,  % change tabs to spaces
  keywordsprefix={@},  % capture method annotations and doctools
  %showtabs=true,
  %tab=|,
  xleftmargin=1cm,
}

% TODO: Modify listings in the next lab so that I don't have to wrap in figure
%       env anymore -- makes the following lines irrelevant
\renewcommand{\lstlistingname}{Figure}

\newcommand{\centeredimage}[2]{%
  \begin{center}
    \includegraphics[scale=#1]{#2}
  \end{center}
}

%% Make a multi-page code figure (only way I've figured out how to do this conveniently as of yet).
% Label, caption, and code path
\newcommand{\iftcodefigure}[3]{%
  \begin{codefigure}
    \label{#1}
    \addtocounter{figure}{-1}
    \lstinputlisting[language=java, 
                     backgroundcolor=\color{background},
                     framexleftmargin=10pt,
                     framextopmargin=6pt,
                     framexbottommargin=6pt,
                     frame=tb,
                     framerule=0pt, 
                     caption={#2}, captionpos=b]{#3}
  \end{codefigure}
}

\makeatletter
% Top-align float-only pages
% https://tex.stackexchange.com/a/28565/96382
\setlength{\@fptop}{0pt}
\setlength{\@fpbot}{0pt plus 1fil}

% Share Figure and listings counter
\AtBeginDocument{%
  \let\c@figure\c@lstlisting
  \let\thefigure\thelstlisting
  \let\ftype@lstlisting\ftype@figure
}
\makeatother

\title{\vspace{6ex}Object-Oriented Programming in Java\\
  \Large IFT 194: Lab 3}
\author{Brandon Doyle\\
\href{mailto:bdoyle@asu.edu}{bdoyle5\at{}asu.edu}\\
1215232174\\[1em]
Dr. Usha Jagannathan\\
\href{mailto:Usha.Jagannathan@asu.edu}{Usha.Jagannathan\at{}asu.edu}}

\setlength{\parindent}{0em}
\setlength{\parskip}{0.5em}
\renewcommand*\contentsname{Summary}
\setcounter{secnumdepth}{0}

\begin{document}
\begin{titlepage}
\clearpage\maketitle
\thispagestyle{empty}
\end{titlepage}
\tableofcontents
\newpage
\blfootnote{View the source of this document on \href{https://github.com/bjd2385/IFT_194_labs/blob/master/\jobname.tex}{GitHub}.}
\section{Prelab Exercises}
\begin{enumerate}
  \item Class constructors are special methods that are called when the \texttt{new} operator is used to create a new instance of a class. These methods typically set up attributes of the class.

        There are a few differences between regular methods and class constructors (special methods). First and foremost, constructors cannot have a return value, and neither is a return type specified in the method header. Secondly, constructors have the same name as the containing class. Thus, if I were to write a class \texttt{Animal}, this class' constructors would also have the name \texttt{Animal}.
  \item Access or visibility modifiers, which include \texttt{public} and \texttt{private}, among others, are used to determine where a variable or method of a class may be accessed from. For example, a public method or field variable may be accessed from outside an instance, whereas in the latter case they may only be used from inside the instance.

        A programmer can decide if a variable should be public based on what that variable may store. Likewise, a programmer may make a method public if it plays a vital role in the class' interface to other classes. Private methods are typically used to provide support to public methods. We should also keep \textit{encapsulation} in mind, which suggests that a class should not allow other classes to modify its state by ``reaching in'' and modifying a value. Rather, we should provide an interface, perhaps through a public method, that has the ability to modify its state. This being said, it is perfectly fine to define publicly-accessible constants as long as they are preceeded by \texttt{final}, declaring the value immutable.

        There are a lot of best-practices suggested throughout the text for this course. I completely support them, knowing that things can easily get out of hand as a code base becomes large enough that a single person can no longer maintain or extend it.
  \item In this problem we consider a class that may represent a bank account.
        \begin{enumerate}[label=\alph*.]
          \item To hold information about an account balance, the name of the account holder, and an account number, I might define the following field variables.
                \begin{lstlisting}[language=java, xleftmargin=0.3\textwidth, backgroundcolor=\color{white}]
private double balance = 0.0;
final private String accountHolder;
final private int[] accountNumber;
                \end{lstlisting}
                The \texttt{accountHolder} and \texttt{accountNumber} fields are declared \texttt{final} because I don't want them to be modifiable once the class is instantiated. Moreover, all variables are private, because I wouldn't want this information to be accessible by other classes or objects unless the proper indentification is provided or process completed.
          \item In this sub-problem, we're asked to write method headers for each of the examples provided.
                \begin{enumerate}[label=\roman*.]
                  \item To withdraw a certain amount of money from the account -- change the account balance and do not return a value.
                    \begin{lstlisting}[language=java, xleftmargin=0.22\textwidth, backgroundcolor=\color{white}]
public void withdraw(double amount) { ... }
                        \end{lstlisting}
                  \item Deposit a certain amount into the account and do not return a value.
                        \begin{lstlisting}[language=java, xleftmargin=0.2\textwidth, backgroundcolor=\color{white}]
public void deposit(double amount) { ... }
                        \end{lstlisting}
                  \item Get the current balance of the account.
                        \begin{lstlisting}[language=java, xleftmargin=0.06\textwidth, backgroundcolor=\color{white}]
public double getBalance(/* something to check credentials? */) { ... }
                        \end{lstlisting}
                  \item Return a string with account information, including name, account number, and balance.
                        \begin{lstlisting}[language=java, xleftmargin=0.05\textwidth, backgroundcolor=\color{white}]
public String getAccountInfo(/* Again, check credentials? */) { ... }
                        \end{lstlisting}
                  \item Charge a \$10 fee.
                        \begin{lstlisting}[language=java, xleftmargin=0.18\textwidth, backgroundcolor=\color{white}]
public void chargeFee(double amount) { ... }
                        \end{lstlisting}
                  \item A constructor that requires the initial balance, name of the owner, and account number.
                        \begin{lstlisting}[language=java, xleftmargin=0.01\textwidth, backgroundcolor=\color{white}]
public BankAccount(double initialBalance, String owner, int[] acctNumber) { ... }
                        \end{lstlisting}
                \end{enumerate}
        \end{enumerate}
\end{enumerate}
\section{Bank Account Class}
\begin{enumerate}
  \item For this question, I've reproduced the code provided in the lab in \hyperref[fig:one]{Figure 1}.
        \begin{enumerate}[label=\alph*.]
          \item Please see my implementation of \texttt{toString} in the aforementioned figure, which overrides \texttt{Object}'s default implementation. Also, I've used the \texttt{String} class' \texttt{format} method, which allows us to shorten this line somewhat, in addition to limiting the display of account balance to 2 decimal places.
          \item For the \texttt{chargeFee} method, I've actually used method overloading so that a default fee of \$10 may be charged to an account if no arguments are supplied.
          \item See the aforementioned figure regarding changes to both \texttt{chargeFee} methods.
          \item See the aforementioned figure for my implementation of the \texttt{changeName} method. My implementation also requires the argument to have content.
        \end{enumerate}
  \item See \hyperref[fig:two]{Figure 2} for my implementation of \texttt{ManageAccounts.java}. The program's output is as follows.
        \begin{verbbox}[\mbox{}\scriptsize]
Joe's new balance: 600.0
New balance: 950.00
Sally's new balance: 950.0

Name: Sally
Acct Number: 0000000001
Balance: 940.00

Name: Joseph
Acct Number: 0000000002
Balance: 565.00
        \end{verbbox}
        \begin{center}
          \theverbbox
        \end{center}
\end{enumerate}
\iftcodefigure{fig:one}{Account.java}{%
  /home/brandon/eclipse-workspace/ift_194_labs/src/lab_3/Account.java}
\iftcodefigure{fig:two}{ManageAccounts.java}{%
  /home/brandon/eclipse-workspace/ift_194_labs/src/lab_3/ManageAccounts.java}
\section{Tracking Grades}
In this section we're tasked with writing a class that permits a teacher to keep track of grades attributed to each student. Please see \hyperref[fig:three]{Figure 3} and \hyperref[fig:four]{Figure 4} for my solution. See below for another example session.
\begin{verbbox}[\mbox{}\scriptsize]
Enter student's grade for test #1: 50
Enter student's grade for test #2: 75
The average for Mary is: 62.50

Enter student's grade for test #1: 75
Enter student's grade for test #2: 50
The average for Mike is: 62.50

Name: Mary
Test 1: 50.00
Test 2: 75.00

Name: Mike
Test 1: 75.00
Test 2: 50.00
\end{verbbox}
\begin{center}
  \theverbbox
\end{center}
\iftcodefigure{fig:three}{Student.java}{%
  /home/brandon/eclipse-workspace/ift_194_labs/src/lab_3/Student.java}
\iftcodefigure{fig:four}{Grades.java}{%
  /home/brandon/eclipse-workspace/ift_194_labs/src/lab_3/Grades.java}
\section{Band Booster Class}
In this exercise we're tasked with writing a class that maintains the state of a band booster and keeps track of band candy sales over 3 weeks. Please see \hyperref[fig:five]{Figure 5} for my solution. Below is another example session, demonstrating how my program works.
\begin{verbbox}[\scriptsize\mbox{}]
Please enter a name: Brandon
Would you like to enter a another booster? [y|n]: yes
Please enter a name: Alison
Would you like to enter a another booster? [y|n]: n

    Week 1

Enter a number of sales for Brandon for this week: 50
Enter a number of sales for Alison for this week: 52

    Week 2

Enter a number of sales for Brandon for this week: 30
Enter a number of sales for Alison for this week: 60

    Week 3

Enter a number of sales for Brandon for this week: 70
Enter a number of sales for Alison for this week: 55

Summary: 

    Brandon: 150
    Alison: 167
\end{verbbox}
\begin{center}
  \theverbbox
\end{center}
\iftcodefigure{fig:five}{BandBooster.java}{%
  /home/brandon/eclipse-workspace/ift_194_labs/src/lab_3/BandBooster.java}
\section{Representing Names}
In this section we're tasked with writing a class that stores a person's first, middle, and last names. Please see \hyperref[fig:six]{Figure 6} and \hyperref[fig:seven]{Figure 7} for my solution. Also, below is a sample session.
\begin{verbbox}[\mbox{}\scriptsize]
Enter a name: Brandon James
*** Error: Please enter a First, Middle, and Last name, separated by space
Enter a name: Brandon James       Doyle
Enter a name: Brandon James Doyle

Brandon James Doyle
Doyle, Brandon James
BJD
17

Brandon James Doyle
Doyle, Brandon James
BJD
17

The names are equal
\end{verbbox}
\begin{center}
  \theverbbox
\end{center}
\iftcodefigure{fig:six}{Name.java}{%
  /home/brandon/eclipse-workspace/ift_194_labs/src/lab_3/Name.java}
\iftcodefigure{fig:seven}{TestNames.java}{%
  /home/brandon/eclipse-workspace/ift_194_labs/src/lab_3/TestNames.java}
\section{Conclusion}
In this lab I came across a number of new challenges. For instance, I wanted to put constraints on incoming values accepted by methods in my classes (hence the \texttt{throws IllegalArgumentException}s). I'm also trying to decide what kind of semantics make sense while I'm designing a class. I try to keep in mind how my class may be used; e.g. how can I make the API more convenient? What kind of type should I use to represent this information (again, what limits do I want to put on the inputs)?
\end{document}
