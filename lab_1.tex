\documentclass[leqno, 11pt]{article}

\usepackage{lmodern}
\usepackage[scaled]{beramono}
\usepackage[T1]{fontenc}
\usepackage{amssymb}
\usepackage{amsmath}
\usepackage{hyperref}
\hypersetup{%
  colorlinks=true,
  linkcolor=magenta,
  filecolor=magenta,
  urlcolor=magenta
}

\usepackage[margin=1in]{geometry}
\usepackage{listings}
\usepackage{graphicx}
\usepackage{caption}

%\graphicspath{"/home/brandon/Desktop/IFT_194/labs/photos"}

\usepackage{xcolor}
\definecolor{javacommentscolor}{HTML}{646464}
\definecolor{javakeywordscolor}{HTML}{7F0055}
\definecolor{javastringscolor}{HTML}{2A00FF}

\lstset{%
  basicstyle=\footnotesize\ttfamily, % code to be displayed as monospace
  breaklines=true,
  %frame=b
  commentstyle=\color{javacommentscolor},
  keywordstyle=\color{javakeywordscolor},
  stringstyle=\color{javastringscolor},
  showstringspaces=false,  % do not show string spaces character
  tabsize=4,  % change tabs to spaces
  keywordsprefix={@},  % capture method annotations and doctools
  %showtabs=true,
  %tab=|
}

\newcommand{\centeredimage}[2]{%
  \begin{center}
    \includegraphics[scale=#1]{#2}
  \end{center}
}

\title{\vspace{6ex}The Java Programming Structure\\
  \Large IFT 194: Lab 1}
\author{Brandon Doyle\\
\href{mailto:bdoyle@asu.edu}{bdoyle5\mbox{}{\fontfamily{ptm}\selectfont @}\mbox{}asu.edu}\\
1215232174\\[1em]
Dr. Usha Jagannathan\\
\href{mailto:Usha.Jagannathan@asu.edu}{Usha.Jagannathan\mbox{}{\fontfamily{ptm}\selectfont @}\mbox{}asu.edu}}

\setlength{\parindent}{0em}
\setlength{\parskip}{0.5em}

\begin{document}
\begin{titlepage}
\clearpage\maketitle
\thispagestyle{empty}
\end{titlepage}

\section*{Part A}
In this activity I didn't learn many new things, but this is primarily 
because I've already taken an introductory course in Java. I was not aware,
however, of the history of Java's versioning (numbering) system, so I appreciated
that background. The objective of this activity is to get students up and running
with an environment tailored to writing Java. The activity also walks us
through the installation process of the Java Developement Kit (JDK) and Eclipse,
an Integrated Development Envirment (IDE) for Java.

I've installed the JDK on my laptop, which is running Ubuntu 16.04 LTS. The process
is quite simple -- all we need to do is download the appropriate JDK file and add the
included \texttt{bin/} subdirectory (wherever it may be) to our path. The \texttt{bin/}
subdirectory contains all the executables for running our code. It's actually quite convenient, because a lot of languages (like  Python) require compilation of some sort. Also, decompressing compressed \texttt{tar} archives can be even simpler for Linux distributions than remembering all of the appropriate flags with \texttt{dtrx}, short for ``do the right extraction.'' The package is written in Python and you can view it on \href{https://github.com/moonpyk/dtrx}{GitHub}.

From here, I can verify the installation as follows in a shell.
\centeredimage{0.6}{photos/jdk.png}
According to Oracle's \href{http://www.oracle.com/technetwork/java/javase/downloads/index.html}{downloads} page, this is the latest version (as of July 2, 2018). Moreover, looking over my \texttt{\textasciitilde/.bashrc}, it appears at some point in the past I added the package to my path.
\centeredimage{0.6}{photos/path.png}
I've also re-created \texttt{Welcome.java} (cf. \autoref{fig:one}) in my eclipse-workspace, and I'm able to run it as follows from my terminal.
\centeredimage{0.6}{photos/first_program.png}
Installing Eclipse is almost as easy in Linux, but I had to add an appropriate \texttt{eclipse.desktop} file under \texttt{/usr/share/applications/} in order to lock the icon to my Launcher.

See \autoref{image:one} for an image of my IDE.
\begin{figure}[h!]
  \centeredimage{0.34}{photos/ide.png}
  \caption{Eclipse Photon.}
  \label{image:one}
\end{figure}

I was also not aware that Java SE 10 had been released. Features of modern Java
that I will appreciate learning more about include more of the functional
capabilities, such as those introduced in Java SE8. I'm also aware of various
projects, like the \href{https://github.com/functionaljava/functionaljava}{functional java} 
library, which extends on many of these capabilities.
\section*{Part B}
\subsection*{1 Poem}
\setcounter{subsection}{1}
\subsection{Poem}
Content

\section*{Conclusion}
I spent approximately 5 hours completing this lab. The quickest portion was
setting up my environment as I already had the JDK installed on my Linux 
machine and Eclipse.

Challenges I faced in writing this lab report were primarily around formatting.
Because I've chosen \LaTeX to present my code and findings, 
\newpage
\begin{figure}[t!]
  \centering
  \lstinputlisting[language=java]{/home/brandon/eclipse-workspace/ift_194_labs/src/lab_1/Welcome.java}
  \caption{Welcome.java}
  \label{fig:one}
\end{figure}

\begin{figure}
  \centering
  \lstinputlisting[language=java]{/home/brandon/eclipse-workspace/ift_194_labs/src/lab_1/Count.java}
  \caption{Count.java}
  \label{fig:two}
\end{figure}

%\begin{figure}
%  \centering
%  \lstinputlisting[language=python]{/home/brandon/projects/visual-inspection-python/example.py}
%  \caption{test\_onto.py}
%  \label{fig:three}
%\end{figure}

\end{document}
