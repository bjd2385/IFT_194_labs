\documentclass[leqno, 11pt]{article}

\usepackage{lmodern}
\usepackage[scaled]{beramono}
\usepackage[T1]{fontenc}
\usepackage{amssymb}
\usepackage{tikz-uml}
\usepackage{amsmath}
\usepackage{hyperref}
\hypersetup{%
  colorlinks=true,
  linkcolor=magenta,
  filecolor=magenta,
  urlcolor=magenta
}

\usepackage[margin=1in]{geometry}
\usepackage{listings}
\usepackage{graphicx}
\usepackage{caption}

\captionsetup{%
  width=1.0\linewidth,
  justification=centering
}

\usepackage{enumitem} % allow characters instead of numbers
\usepackage{verbatimbox}  % center \verbatim env in figure env
\usepackage{siunitx}  % SI unit formatting
\usepackage{colortbl}  % color tabular rows/columns
\usepackage[most]{tcolorbox}

% TODO: Set up captions with tcolorbox so the user doesn't have to explicitly
%       pass a caption={} to each \lstlisting command.
\newtcolorbox[blend into=figures]{codefigure}[1][]{%
  enhanced jigsaw, 
  float=!ht, 
  breakable, 
  frame hidden, 
  colback=white,
  #1
}

% Better `@' symbol
\newcommand{\at}{\mbox{}{\fontfamily{ptm}\selectfont @}\mbox{}}

\newcommand\blfootnote[1]{%
  \begingroup
    \renewcommand\thefootnote{}\footnote{#1}
    \addtocounter{footnote}{-1}
  \endgroup
}

%\graphicspath{"/home/brandon/Desktop/IFT_194/labs/photos"}

\usepackage{xcolor}

\definecolor{tableheaderrow}{HTML}{CED3DB}

\definecolor{javacommentscolor}{HTML}{646464}
\definecolor{javakeywordscolor}{HTML}{7F0055}
\definecolor{javastringscolor}{HTML}{2A00FF}

\definecolor{background}{HTML}{F2F7FF}

\lstset{%
  basicstyle=\linespread{0.8}\scriptsize\ttfamily, % code to be displayed as monospace
  breaklines=true,
  %frame=b
  commentstyle=\color{javacommentscolor},
  keywordstyle=\color{javakeywordscolor},
  stringstyle=\color{javastringscolor},
  showstringspaces=false,  % do not show string spaces character
  tabsize=4,  % change tabs to spaces
  keywordsprefix={@},  % capture method annotations and doctools
  %showtabs=true,
  %tab=|,
  xleftmargin=1cm,
  morekeywords={var}
}

% TODO: Modify listings in the next lab so that I don't have to wrap in figure
%       env anymore -- makes the following lines irrelevant
\renewcommand{\lstlistingname}{Figure}

\newcommand{\centeredimage}[2]{%
  \begin{center}
    \includegraphics[scale=#1]{#2}
  \end{center}
}

%% Make a multi-page code figure (only way I've figured out how to do this conveniently as of yet).
% Label, caption, and code path
\newcommand{\iftcodefigure}[3]{%
  \begin{codefigure}
    \label{#1}
    \addtocounter{figure}{-1}
    \lstinputlisting[language=java,
                     morekeywords={var},
                     backgroundcolor=\color{background},
                     framexleftmargin=10pt,
                     framextopmargin=6pt,
                     framexbottommargin=6pt,
                     frame=tb,
                     framerule=0pt, 
                     caption={#2}, captionpos=b]{#3}
  \end{codefigure}
}

\makeatletter
% Top-align float-only pages
% https://tex.stackexchange.com/a/28565/96382
\setlength{\@fptop}{0pt}
\setlength{\@fpbot}{0pt plus 1fil}

% Set up the table of contents so that titles are displayed
\renewcommand*\contentsname{Summary}
\setcounter{secnumdepth}{0}

% Share Figure and listings counter
\AtBeginDocument{%
  \let\c@figure\c@lstlisting
  \let\thefigure\thelstlisting
  \let\ftype@lstlisting\ftype@figure
}
\makeatother

\title{\vspace{6ex}Exception Handling in Java\\
  \Large IFT 194: Lab 6}
\author{Brandon Doyle\\
\href{mailto:bdoyle@asu.edu}{bdoyle5\at{}asu.edu}\\
1215232174\\[1em]
Dr. Usha Jagannathan\\
\href{mailto:Usha.Jagannathan@asu.edu}{Usha.Jagannathan\at{}asu.edu}}

\setlength{\parindent}{0em}
\setlength{\parskip}{0.5em}

\begin{document}
\begin{titlepage}
\clearpage\maketitle
\thispagestyle{empty}
\end{titlepage}
\tableofcontents
\blfootnote{View the source of this document on \href{https://github.com/bjd2385/IFT_194_labs/blob/master/\jobname.tex}{GitHub}.}
\newpage
\section{Exceptions aren't always errors}
For this section we're given a class \texttt{CountLetters} (cf. \hyperref[fig:one]{Figure 1}) that reads a word from the user and prints the number of occurrences of each letter in the word. However, we're of course not guaranteed to \textit{only} get letters, so it will throw an error if we provide any other input. 
\iftcodefigure{fig:one}{CountLetters.java}{%
  /mnt/c/Users/bdoyle/eclipse-workspace/ift_labs/src/lab_6/CountLetters.java}\\
You may also uncomment the loop containing the call to \texttt{scnr.nextLine}, which uses a regular expression to ensure the input contains only a certain set of characters.
\section{Placing Exception Handlers}
In this section we're given a class \texttt{ParseInts} in \hyperref[fig:two]{Figure 2} for parsing a line of text containing integers (hopefully).
\iftcodefigure{fig:two}{ParseInts.java}{%
  /mnt/c/Users/bdoyle/eclipse-workspace/ift_labs/src/lab_6/ParseInts.java}\\
However, as it stands this program is extremely frail. For example, if I input a string ``\texttt{20 56}'', we receive the following output.
\begin{verbbox}[\mbox{}\scriptsize]
Enter a line of text: 20 56
The sum of the integers on this line is: 76
\end{verbbox}
\begin{center}
  \theverbbox
\end{center}
On the other hand, adding a single space to the beginning of the input, let alone a character that is not a number, produces a disastrous result.
\begin{verbbox}[\mbox{}\scriptsize]
Enter a line of text:  55 66
Exception in thread "main" java.lang.NumberFormatException: For input string: ""
  at java.base/java.lang.NumberFormatException.forInputString(Unknown Source)
  at java.base/java.lang.Integer.parseInt(Unknown Source)
  at java.base/java.lang.Integer.parseInt(Unknown Source)
  at ift_labs/lab_6.ParseInts.main(ParseInts.java:18)
\end{verbbox}
\begin{center}
  \theverbbox
\end{center}
One way to fix this error is to encapsulate the loop in a \texttt{try} block, as in \hyperref[fig:three]{Figure 3}.
\iftcodefigure{fig:three}{ParseIntsBetter.java}{%
  /mnt/c/Users/bdoyle/eclipse-workspace/ift_labs/src/lab_6/ParseIntsBetter.java}\\
\section{Throwing Exceptions}
In this section we're given the two classes in \hyperref[fig:four]{Figure 4}, which I've rewritten to compute factorials of input integers \textit{safely}. Below is an example session with the program.
\begin{verbbox}[\mbox{}\scriptsize]
Enter an integer: hello!
*** Error: please enter an integer
Enter an integer: 15
15! = 2004310016
Compute another factorial? [y|n]: y
Enter an integer: 13
13! = 1932053504
Compute another factorial? [y|n]: n
Exiting.
\end{verbbox}
\begin{center}
  \theverbbox
\end{center}
\section{Conclusion}
This has been a great experience!

I spent approximately 3 hours cumulatively writing this lab. I learned about writing and throwing exceptions in my own methods. I also think it would be very interested writing our own exceptions to be thrown in our packages. 

I also think that it would be useful learning more about text processing, so that we don't have to throw exceptions every time some data doesn't look as it's supposed to.

A feature of pure functional languages like Haskell that I really admire is \textit{totality}. The most important part of \href{https://en.wikipedia.org/wiki/Total_functional_programming}{totality} is, in my opinion, the fact that functions are guaranteed to have an output for every input. In reality/practice this can be extremely difficult to guarantee; for instance, even while computing something as fundamental as a continuous factorial (cf. the \href{https://en.wikipedia.org/wiki/Gamma_function}{gamma function}), we can run into a plethora of issues.
\iftcodefigure{fig:four}{Factorials.java}{%
  /mnt/c/Users/bdoyle/eclipse-workspace/ift_labs/src/lab_6/Factorials.java}
\end{document}
